\documentclass[italian,12pt,a4paper]{article}
\usepackage[utf8]{inputenc}
\usepackage[T1]{fontenc}
\usepackage{mathtools}
\usepackage{blkarray, bigstrut} %
\usepackage{babel}
\usepackage{graphicx}
\usepackage{subfig}
\usepackage{hyperref}
\usepackage{tikz}
\usepackage{colortbl}
\usepackage{pgf-pie}
\usepackage{algorithm}
\usepackage{algpseudocode}
\usepackage{algorithmicx}
\usepackage{placeins}
\usepackage{fancyvrb}
\usepackage{svg}
\usepackage{tabularx}
\title{Università degli studi di Bari facoltà di scienze MM.FF.NN}
\date{} % clear date
\hypersetup{
	colorlinks=true,
	linkcolor=black,
	filecolor=magenta,      
	urlcolor=cyan,
	pdfpagemode=FullScreen,
}
\graphicspath{ {./img/} }
\RequirePackage[subfigure]{tocloft}

\cftsetindents{section}{0em}{2em}
\cftsetindents{subsection}{0em}{2em}

\renewcommand\cfttoctitlefont{\hfill\Large\bfseries}
\renewcommand\cftaftertoctitle{\hfill\mbox{}}

\algrenewcommand\algorithmicrequire{\textbf{Input:}}
\algrenewcommand\algorithmicensure{\textbf{Output:}}


\setcounter{tocdepth}{2}
\begin{document}
	\maketitle
	\thispagestyle{empty}
	\begin{center}
		\huge	\textbf{Progetto ingegneria della conoscenza}
		\linebreak
		\linebreak
		\Large \textbf{TrainDelay-project}
	\end{center}
	
	
	
	\begin{center}
		by \\
		\Large \textbf{Vito Proscia mat. 735975}
	\end{center}

	
	\begin{figure}[hb]
		\centering
		\includegraphics[width=5cm]{image.png}
	\end{figure}
	
	\vfill
	\begin{center}
		Anno accadenico 2022-2023
	\end{center}
	
	\newpage
	
	\tableofcontents

	\newpage

	
	\section{Introduzione}

	\subsection{Contesto}
	
	TrainDelay-project è 

	
	\subsection{Definizione obiettivo principale}
	
	L'obiettivo principale del progetto è la creazione di un motore di ricerca che trova i migliori itinerari di viaggio in treno sulla base della stazione di partenza e di arrivo e che, per ogni viaggio, mostra una predizione del probabile ritardo.\\
	Questo sistema non solo potrà far risparmiare del tempo a chi organizza dei viaggi valutando ogni singola tratta in tremini di stazioni e orari di partenza e di arrivo, ma garantirà un risparmio econimico ai viaggiatori garantendo che la tratta scelta dal sistema sia la minima e necessaria per arrivare alla destinazione. 
	
	
	\subsection{Tool utilizzati}
	Per la sperimentazione sono stati usati diversi stumenti/librerie, quali:
	
		\begin{itemize}
			\item \textbf{PySWIP}, libreria Python che fornisce un'interfaccia per utilizzare SWI-Prolog, usato per la rappreesentazione formale della schedul dei treni
			\item \textbf{NetworkX}, libreria Python utilizzata per la creazione, l'analisi e la manipolazione di reti complesse. Questa libreria fornisce un insieme di strumenti per la rappresentazione di reti e grafi, oltre a un'ampia gamma di algoritmi e funzioni per eseguire diverse operazioni su di essi.
		\end{itemize}

	\section{Rappresentazione formale della conoscenza}
	La parte iniziale del progetto si è concentarta sulla rappresentazione formale attraverso fatti e regole Prolog della schedul dei treni e delle stazioni, in particolare ogni treno è caratterizzato da:
	
		\begin{enumerate}
			\item ID treno, identificatore univoco del treno
			\item Tipo di treno, regionale o nazionale
			\item ID stazione di partenza
			\item ID stazione di arrivo
			\item Orario di partenza (HH:MM)
			\item Orario di arrivo (HH:MM)
			\item Lista delle fermate
		\end{enumerate}
		Esempio:
		
\begin{small}

\begin{verbatim}
	train(320, nazionale, s01700, s01301, "15:10", "15:58", [s01700, s01307, s01301]).
	train(321, nazionale, s01301, s01700, "18:02", "18:50", [s01301, s01307, s01700]).
	train(322, nazionale, s01700, s01301, "17:10", "17:58", [s01700, s01307, s01301]).
	train(323, nazionale, s01301, s01700, "20:02", "20:50", [s01301, s01307, s01700]).
	train(324, nazionale, s01700, s01301, "19:10", "19:58", [s01700, s01307, s01301]).
	train(325, nazionale, s01301, s01700, "22:02", "22:50", [s01301, s01307, s01700]).
\end{verbatim}
\end{small}
	Mentre ogni stazione è caratterizzata da:
	
			\begin{enumerate}
				\item ID stazione, identificatore univoco delle stazioni
				\item Nome stazione
				\item Regione stazione
			\end{enumerate}
		Esempio:

\begin{small}
	
	\begin{verbatim}
		station(s11504, "ACQUAVIVA DELLE FONTI", "Puglia").
		station(s12026, "ACQUEDOLCI-S.FRATELLO", "Sicilia").
		station(s00867, "ACQUI TERME", "Piemonte").
		station(s11907, "ACRI BISIGNANO LUZZI", "Calabria").
		station(s05420, "ADRIA", "Veneto").
	\end{verbatim}
\end{small}
	
	Tutte le informazioni relative


	\section{Analisi del dataset}
	
	\subsection{Descrizione features}

	
	\subsection{Preparazione dati}
	
	\subsubsection{Analisi delle input features}

	
	\subsubsection{Analisi della target feature}

	
	\section{Machine Learning}
	
	\section{Valutazione dei modelli}

	
	\subsection{Metriche scelte}


	\section{Risultati}


	
	\subsubsection{Considerazioni}
	
	\section{Sviluppi futuri}
	Il sistema presentato è aperto a sviluppi futuri che possano rendere il sistema ancora più efficiente e all'avanguardia. \\
	Di seguito sono descritti alcuni dei possibili sviluppi futuri che intendiamo esplorare:
	
	\begin{enumerate}
		\item Espansione della copertura ferroviaria andando a completare le informazioni relative a treni e staziono ed integrazione con altri servizi ferroviari (Italo, Frecciarossa, ...)
		\item Implementazione di una vera e propria interfaccia grafica
		\item Miglioramento della ricerca andando a suggerire all'utente, sulla base dei caratteri inseriti, le stazioni che iniziano con quei caratteri
		\item Integrazione di dati in tempo reale andando a fornire all'utente informazioni in \textit{real time}

	\end{enumerate}
	
	
\end{document}