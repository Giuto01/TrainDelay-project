\documentclass[italian,12pt,a4paper]{article}
\usepackage[utf8]{inputenc}
\usepackage[T1]{fontenc}
\usepackage{mathtools}
\usepackage{blkarray, bigstrut} %
\usepackage{babel}
\usepackage{graphicx}
\usepackage{subfig}
\usepackage{hyperref}
\usepackage{tikz}
\usepackage{colortbl}
\usepackage{pgf-pie}
\usepackage{algorithm}
\usepackage{algpseudocode}
\usepackage{algorithmicx}
\usepackage{placeins}
\usepackage{svg}
\usepackage{tabularx}
\title{Università degli studi di Bari facoltà di scienze MM.FF.NN}
\date{} % clear date
\hypersetup{
	colorlinks=true,
	linkcolor=black,
	filecolor=magenta,      
	urlcolor=cyan,
	pdfpagemode=FullScreen,
}
\graphicspath{ {./img/} }
\RequirePackage[subfigure]{tocloft}

\cftsetindents{section}{0em}{2em}
\cftsetindents{subsection}{0em}{2em}

\renewcommand\cfttoctitlefont{\hfill\Large\bfseries}
\renewcommand\cftaftertoctitle{\hfill\mbox{}}

\algrenewcommand\algorithmicrequire{\textbf{Input:}}
\algrenewcommand\algorithmicensure{\textbf{Output:}}

\setcounter{tocdepth}{2}
\begin{document}
	\maketitle
	\thispagestyle{empty}
	\begin{center}
		\huge	\textbf{Progetto ingegneria della conoscenza}
		\linebreak
		\linebreak
		\Large \textbf{TRENITALIA - TrainDelay-project}
	\end{center}
	
	
	
	\begin{center}
		by \\
		\Large \textbf{Vito Proscia mat. 735975}
	\end{center}

	
	\begin{figure}[hb]
		\centering
		\includegraphics[width=5cm]{image.png}
	\end{figure}
	
	\vfill
	\begin{center}
		Anno accadenico 2022-2023
	\end{center}
	
	\newpage
	
	\tableofcontents

	\newpage

	
	\section{Introduzione}

	\subsection{Contesto}

	
	\subsection{Definizione obiettivo principale}

	
	\subsection{Metodologia}
.
	
	
	\subsection{Tool utilizzati}
	Per la sperimentazione sono stati usati diversi stumenti, quali:


	\section{Analisi del dataset}
	
	\subsection{Descrizione features}

	
	\subsection{Preparazione dati}
	
	\subsubsection{Analisi delle input features}

	
	\subsubsection{Analisi della target feature}

	
	\section{Machine Learning}
	
	\section{Valutazione dei modelli}

	
	\subsection{Metriche scelte}


	\section{Risultati}


	
	\subsubsection{Considerazioni}
	
	
\end{document}